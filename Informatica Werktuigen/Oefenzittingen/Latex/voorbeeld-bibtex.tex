\documentclass[a4paper]{article}
\usepackage[dutch]{babel}
\usepackage[final]{graphicx}
\usepackage{url}

\begin{document}
\title{String Theory}
\maketitle

\section{Introduction}

String theory is a developing branch of theoretical physics that
combines quantum mechanics and general relativity into a quantum
theory of gravity \cite{Mukhi:1999:introduction}. The strings of
string theory are one-dimensional oscillating lines, but they are no
longer considered fundamental to the theory, which can be formulated
in terms of points or surfaces too. Since its inception as the dual
resonance model which described the strongly interacting hadrons as
strings, the term string theory has changed to include any of a
group of related superstring theories which unite them. One shared
property of all these theories is the holographic principle. String
theory itself comes in many different formulations, each one with a
different mathematical structure, and each best describing different
physical circumstances. But the principles shared by these
approaches, their mutual logical consistency, and the fact that some
of them easily include the standard model of particle physics, has
led many physicists to believe that the theory is the correct
fundamental description of nature. In particular, string theory is
the first candidate for the theory of everything (TOE), a way to
describe the known fundamental forces (gravitational,
electromagnetic, weak and strong interactions) and matter (quarks
and leptons) in a mathematically complete system. Many detractors
criticize string theory as it has not provided quantitative
experimental predictions. Like any other quantum theory of gravity,
it is widely believed that testing the theory directly would require
prohibitively expensive feats of engineering. Whether there are
stringent indirect tests of the theory is unknown. String theory is
of interest to many physicists because it requires new mathematical
and physical ideas to mesh together its very different mathematical
formulations. One of the most inclusive of these is the
11-dimensional M-theory, which requires spacetime to have eleven
dimensions \cite{Duff:1995:origin}, as opposed to the usual three
spatial dimensions and the fourth dimension of time. The original
string theories from the 1980s describe special cases of M-theory
where the eleventh dimension is a very small circle or a line, and
if these formulations are considered as fundamental, then string
theory requires ten dimensions. But the theory also describes
universes like ours, with four observable spacetime dimensions, as
well as universes with up to 10 flat space dimensions, and also
cases where the position in some of the dimensions is not described
by a real number, but by a completely different type of mathematical
quantity. So the notion of spacetime dimension is not fixed in
string theory: it is best thought of as different in different
circumstances \cite{Polchinski:1998:stringtheory}.

\bibliographystyle{plain}
\bibliography{voorbeeld}

\end{document}
