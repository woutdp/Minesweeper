\chapter{De deelnemers}

Tijdens het weekend van 2 en 3 mei 1998 werden de deelnemers aan de
eenheidsmunt geselecteerd. Het zijn volgende landen die vanaf 1
januari 1999 de euro invoeren: Belgi\"e, Nederland, Luxemburg,
Duitsland, Frankrijk, Itali\"e, Spanje, Portugal, Oostenrijk, Ierland
en Finland. Dit zijn de ``ins.'' Deze landen hebben ook een
``stabiliteitspact'' afgesloten waarin ze zich engageren hun
economie\"en blijvend op elkaar af te stemmen.

Vier andere landen die behoren tot de EU nemen niet deel van bij de
start. Het zijn Denemarken, en het Verenigd Koninkrijk die
vrijwillig afzagen van een deelname en Zweden en Griekenland, twee
landen dat niet aan de criteria voldeden. Deze vier landen worden de
``outs.'' geheten. Deze landen kunnen eventueel in een latere fase
deelnemen. Er is hiervoor een tweejaarlijkse evaluatie voorzien,
maar vroegere toetreding is mogelijk op vraag van de kandidaat
lidstaat.